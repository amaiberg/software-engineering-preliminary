
\section{Inheriting from AbstractKeytermExtractor or AbstractKeytermUpdater}

To migrate each of your team members' components, you can do the following:

\begin{enumerate}

\item As suggested earlier, you should create your own components in the package
\texttt{java/edu/cmu/lti/oaqa/openqa/test/teamXX/keyterm}, in case another
\texttt{teamYY} might have a component with exactly the same class name as
yours.

\item \texttt{AbstractKeytermExtractor} is mandatory for the pipeline, which
identifies the keyterms from a question. Most of the time, you might want to
inherit from \texttt{AbstractKeytermExtractor} for your own
\texttt{KeytermExtractor} (please think of a better name for your component
rather than \texttt{KeytermExtractor}, \texttt{MyKeytermExtractor}, or
\texttt{TeamXXNKeytermExtractor}).

In Listing \ref{lst:abstract-keyterm-extractor}, in you can see the only method
your component needs to implement (the hook method) is \verb|getKeyterms|, and
the \verb|process| method helps you retrieve data you need from CAS and store
your output back to CAS.

\lstinputlisting[language=Java,float,linewidth=1.1\textwidth,caption=AbstractKeytermExtractor.java,label=lst:abstract-keyterm-extractor]{AbstractKeytermExtractor.java}

\item Since your component extends \verb|JCasAnnotator_ImplBase|, and
\verb|process| is the only method finalized by the abstract class, which means
if you can still \verb|getConfigParameterValue| within an \verb|@Override|-ing
\verb|initialize| method. But remember to call the parent's \verb|initialize|
method before look up your own configuration parameters. Specifically, you can
write the \verb|initialize| method similar to this:

\lstinputlisting[language=Java,float,linewidth=1.1\textwidth,caption=Sample initialze method,label=lst:initialize]{initialize.java}

\item \verb|Keyterm| is defined in the \textbf{BaseQA} project, and it is easy
to create by using the constructor \verb|Keyterm(String text)|.

\item You can also consider to use the simple keyterm extractor from
\texttt{HelloQA}, and make necessary change to the extracted keyterms with one
or several components that extend \texttt{AbstractKeytermUpdater}. In Listing
\ref{lst:abstract-keyterm-updater}, you can see the only abstract method is
\verb|updateKeyterms|, where you can get all the extract keyterms from a
previous step as inputs, and you have to return another list of keyterms.

\lstinputlisting[language=Java,float,linewidth=1.1\textwidth,caption=AbstractKeytermUpdater.java,label=lst:abstract-keyterm-updater]{AbstractKeytermUpdater.java}

\end{enumerate}
