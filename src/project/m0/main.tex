% 
% This work is licensed under the Creative Commons Attribution-ShareAlike 3.0
% Unported License. To view a copy of this license, visit
% http://creativecommons.org/licenses/by-sa/3.0/.
%
% author: Zi Yang <ziy@cs.cmu.edu>
% date: 09/02/2012
%
\documentclass[oneside]{memoir}
\renewcommand{\chaptername}{Task}

\usepackage{times}

\usepackage{titlesec}
\titleformat{\section}{\normalfont\Large\bfseries}{Task \thesection}{1em}{}

\usepackage{url}
\usepackage{hyperref}

\usepackage{graphicx}
\graphicspath{{../../fig/cse-and-qa/}}

\usepackage{pstricks}
\usepackage{epsfig}

\usepackage{listings}
\usepackage{color}
\definecolor{dkgreen}{rgb}{0,0.6,0}
\definecolor{gray}{rgb}{0.5,0.5,0.5}
\definecolor{lightblue}{rgb}{0.95,0.95,1}
\definecolor{mauve}{rgb}{0.58,0,0.82}
\lstset{
  basicstyle=\ttfamily\small,
  numbers=left,
  numberstyle=\tiny\color{gray},
  stepnumber=2,    
  numbersep=5pt,
  backgroundcolor=\color{lightblue},
  showspaces=false,
  showstringspaces=false,
  showtabs=false,
  frame=lines,
  rulecolor=\color{black},
  tabsize=2,
  captionpos=b,
  breaklines=true,
  breakatwhitespace=false,
  title=\lstname,
  keywordstyle=\color{blue},
  commentstyle=\color{dkgreen},
  stringstyle=\color{mauve},
  escapeinside={\%*}{*)},
  morekeywords={*,...},
}
\usepackage{letltxmacro}
\makeatletter
\LetLtxMacro\@@lst@inputlisting\lst@inputlisting
\renewcommand\lst@inputlisting[2][]{%
  \try@listingspath{#2}%
  \if@tempswa
    \typeout{Using \@foundlisting}%
    \@@lst@inputlisting[#1]{\@foundlisting}%
  \else
    \typeout{Missing file #2}\endgroup
  \fi}
\def\listingspath#1{\def\@listingspath{#1}}
\listingspath{}
\def\try@listingspath#1{%
  \@tempswafalse
  \expandafter\@tfor\expandafter\next
  \expandafter:\expandafter=\@listingspath\do
  {\if@tempswa\@break@tfor\fi
   \IfFileExists{\next/#1}{\@tempswatrue\xdef\@foundlisting{\next/#1}}{}}%
}
\makeatletter
\listingspath{{../../lst/cse-and-qa/}}

\usepackage{multirow}

\definecolor{shadecolor}{gray}{0.9}

\newenvironment{qa}
{\begin{shaded}\begin{itemize}}
{\end{itemize}\end{shaded}}

\title{{\bfseries 11-791 Design and Engineering of Intelligent Information
System Fall 2014\\Project}\\
\vspace{1em}
\itshape\rmfamily Building a Pipeline for Biomedical Question
Answering}

\date{}

\begin{document}

\begin{titlingpage}
\maketitle

\hspace{-0.1\textwidth}
\begin{minipage}{1.2\textwidth}
\vspace{-5em}
\textbf{Important dates}
\begin{itemize}

\item \textbf{Hand out: October 24.}\footnote{This version was built on \today}


\item \textbf{Milestone 0 (M0): Project Wiki and proposal draft. Turn in: October 27.}
\begin{itemize}
\item You are required to fill out the project proposal for your assigned team on a newly created GitHub repository. Please send us the URL of your project repository page (i.e.,\url{https://github.com/COURSENUM-ID/project-teamID} where `COURSENUM' is the course designation number 11791/11693, and ID is your assigned team number). We expect to find a filled out draft of the proposal template on your project Wiki. Please note that this is only a draft of your proposal, the deadline is just to get your project wiki and your teams together. You will continue working on this proposal until Milestone 1. Please see the proposal outline at the end of this document.
\end{itemize}
\item \textbf{Milestone 1 (M1) : Concepts, documents, and triples retrieval.  Turn in: November 10.}

\begin{itemize}

\item As in previous homeworks, you are required to submit all the sources for your components via a Maven release, and you DON'T need to submit any other documents. 

\item In addition, please send us the URL of your project repository page (i.e.,\url{https://github.com/COURSENUM-ID/project-teamID} where `COURSENUM' is the course designation number 11791/11693, and ID is your assigned team number). We will look into your Issues page and Wiki page, and expect you have created milestones and issues, reported the results of your components, and completed proposal.

\end{itemize}

\item \textbf{Milestone 2 (M2): Snippets retrieval. Turn in: November 19.}

\begin{itemize}

\item You are required to submit all the sources for your components via a Maven release, and you DON'T need to submit any other documents. 

\item We will again look into your Issues page and Wiki page, and expect you
properly solved your previous created milestones and issues, and probably
created new milestones and issues, and reported the evaluation results of your
M2 on your own, your team meeting minutes and probably your revised project
goals.

\end{itemize}

\end{itemize}

\end{minipage}
\hspace{-0.1\textwidth}

\hspace{-0.1\textwidth}
\begin{minipage}{1.2\textwidth}

\begin{itemize}

\item \textbf{Milestone 3 (M3): Exact answer generation. Turn in: December 1.}

\begin{itemize} 

\item You are required to submit all the sources for your components via a Maven release.

\item We expect you properly solved all the issues and accomplish all
milestones, and reported the evaluation results of your final system, your team
meeting minutes and a final summary.

\item Name your presentation slides as
\texttt{project-teamID.ppt}\footnote{Or other formats, e.g., pptx, odp, pdf, etc.}
and put it in the \texttt{src/main/resources/docs}.

\end{itemize}

\end{itemize}

\end{minipage}
\hspace{-0.1\textwidth} 

\hspace{-0.1\textwidth}
\begin{minipage}{1.2\textwidth}


\begin{comment}
You should always organize your project in the same hierarchy as shown
below for all your submissions:

\small
\begin{verbatim}
hw2-teamXX
|- pom.xml
|- launches
|  `- hellobioqa.launch
|- data                           /* git-ignore this folder, since
|  `- oaqa-eval.db3                * it will become huge */
`- src
     `- main
        |- java/edu/cmu/lti/oaqa/openqa/test/teamXX
      |   `- **/*.java             /* your Java codes go into this 
      |                            * folder as you did before */
        `- resources
           |- input
           |- gs
           `- hellobioqa
              |- collection        /* descriptors for collection reader,
              |- keyterm            * example keyterm extractors,
              |- retrieval          * example retrieval strategists,
              |- passage            * example passage extractors */
              |- hellobioqa.yaml   /* the entry point for the example 
              |                     * pipeline and your pipeline */
              `- teamXX            /* your descriptors go into here */
                 `- **/*.yaml

\end{verbatim}
\begin{comment}
\normalsize

\begin{comment}
Several notes about organizing your Maven project and other additional
information:

\begin{enumerate}

\item \textbf{Submission:} The same way as you did for Homework 0 and 1 (set up
GitHub repo, create Maven project, write your code, submit to Maven repo),
except that the name has changed to hw2-teamXX (XX is a two-digit number from 01
to 18).

\item \textbf{Your source files and descriptors:}
\texttt{java/edu/cmu/lti/oaqa/openqa/test/teamXX} means you should create
directories (or packages in terms of Java program structure) iteratively from
\texttt{java}, all the way to \texttt{teamXX}.

\verb|**/*.java| and \verb|**/*.yaml| tell you that you don't need to flatten
your folder hierarchy, instead we encourage you to place your Java codes in the
right package, and similarly, you can create subfolders for different types of
descriptors, e.g., \verb|src/main/resources/hellobioqa/teamXX/keyterm| for all
the descriptors related to keyterm extraction task.

Actually, once you specify the main yaml descriptor of pipeline to the
\texttt{ECDDriver}, you can have your own entry point to the system. However, if
you want to use \texttt{launches/hellobioqa.launch} file to run the pipeline,
you should consider \texttt{src/main/resources/hellobioqa/hellobioqa.yaml} as
your main yaml. When we evaluate your codes, we will run a different main yaml
to bundle all your components, all the components declared in your
\texttt{hellobioqa.yaml} will be used.
\item \textbf{Comments, Javadocs, and documentations:} They are still important
if you want us and other users to better understand your code. (Remember: we
will become your customer when we run the big experiment.)

\end{enumerate}
\end{comment}
\end{minipage}
\hspace{-0.1\textwidth}

\hspace{-0.1\textwidth}
\begin{minipage}{1.2\textwidth}
 
\textbf{Useful information}
\begin{enumerate}

\item Please visit Piazza regularly to check if a newer version is published. We may have new versions or revised instruction at the begining of
each milestone.

We expect that most of the general communication between the instructor team and students will take place on Piazza
\url{https://piazza.com/class/hyvsubeilei6dd}.
For private questions, e.g., regarding grades, you may contact instructors by e-mail.
Your friendly TAs are:
Avner Maiberg (\href{mailto:amaiberg@andrew.cmu.edu}{\nolinkurl{amaiberg@andrew.cmu.edu}}), 
Parag Argawal (\href{mailto:paraga@andrew.cmu.edu}{\nolinkurl{paraga@andrew.cmu.edu}}), 
Leonid (Leo) Boytsov (\href{mailto:srchvrs@cmu.edu}{\nolinkurl{srchvrs@cmu.edu}}), and 
Xuezi (Manfred) Zhang (\href{mailto:xueziz@andrew.cmu.edu}{\nolinkurl{xueziz@andrew.cmu.edu}}).
\item Again, both source files and pdf file of this assignment are
publicly available on my GitHub
     
\url{http://github.com/amaiberg/software-engineering-preliminary}

Please feel free to fork the project and send a pull request back to me as some
of you did for Homework 0 for any error. Or you can just report an issue at

\url{http://github.com/amaiberg/software-engineering-preliminary/issues}

\end{enumerate}

\end{minipage}
\hspace{-0.1\textwidth}

\end{titlingpage}


\chapter{Implementing A Simple Information Processing Task with UIMA SDK}

In this task, you need to 
create several analysis engines (as well as
respective implementations of annotators) 
and combine them in a single aggregate analysis engine.
You may find \href{http://uima.apache.org/documentation.html#manuals_and_guides}{Apache UIMA Manuals and
Guides}
helpful as they provide step-by-step guidelines for analysis engines creation.


\section{Creating Maven project from the archetype}

For this task, we have prepared another archetype to help you quickly build your
project. The tutorial for Homework 1 might help you create a Maven project from
an archetype.

For Homework 2, you need to add the following Catalog URL

\begin{center}
\textbf{https://raw.githubusercontent.com/amaiberg/DEIIS-hw2-archetype/master/archetype-catalog.xml}
\end{center}

The archetype for Homework is

\begin{center}
\textbf{hw2-archetype}
\end{center}

Also remember that the \textbf{Group Id} and \textbf{Artifact Id} for Homework 2
are

\begin{center}
\textbf{edu.cmu.lti.11791.f14.hw2}
\end{center}

and

\begin{center}
\textbf{hw2-ID}
\end{center}

with ID being your Andrew Id.

Similarly, you need to edit the \texttt{pom.xml} file to provide the SCM
information related to your GitHub repository (it should be different
from the previous homework). 

Unlike Homework 1, we provide a standard type system that you should use.
This type system is represented by an XML file (\texttt{deiis\_types.xml}).
In addition, we provide Java classes generated from this
file (using \texttt{JCasGen}).



\section{Adding descriptors and code for your name entity recognizer}

If you have one or several idea(s) for the Gene Mention tagging task already,
it's time to implement all of them in this task. You will find the official
tutorial is helpful for you to accomplish this task. Except that you are
required to build your component based on UIMA framework, there is little
limitation on how you should design and implement your type system as well as
all the pipeline components. Nevertheless, there are still some suggestions (or
you can say tips or our expectations) to consider.

\subsection{Type system}

\begin{enumerate}

\item The minimal type system to accomplish this task should include types for
input and output elements, i.e. the id and text of each sentence and gene tags.

\item If you intend to design a pipeline of two or more phases (e.g., each of
the first few phases extracts a certain feature, and the last phase predicts the
gene mentions based on all the features), you should include all the
intermediate types in your type system as well.

\item If your type system is complex, then you can try to refactor the type
system by categorizing the types according to their functions (e.g., NLP
related, biological related, etc.), and creating multiple type systems for each
of the categories.

\item Another common practise in the type system design is to create a
\emph{base annotation} type which includes two features: a string feature
\emph{source} and a numeric feature \emph{confidence} to help keep track of
where an annotation was originally made by, and how confidence the annotation
was. All other types then inherit from this base annotation type.

\item You could assign a (or several) name space(s) for your types, e.g., instead
of \texttt{Sentence} (default name space), you could name it
\texttt{model.Sentence} (a type with name space \texttt{model}). Once you click
the \textbf{JCasGen} button, you will find a nice hierarchical folder structure
created under \texttt{src/main/java} for the Java classes corresponding to the
types.

\end{enumerate}

\subsection{Collection processing engine}

\begin{enumerate}

\item You can use UML to help you and us clarify your pipeline design. You could
include some UML diagrams in the Javadocs and in your report.

\item You can refer to the \texttt{SimpeRunCPE.java} in the uima-example package
(and we also included it in the archetype) as the entry point to test your
pipeline once you have done all the implementations.

\item The UIMA Eclipse plug-in has not provided a GUI for you to specify the CPE
descriptor yet, which means although it will not be complicated, you still need to
manually create the CPE descriptor all by yourself.

\item \textbf{Please name your CPE descriptor as CpeDescriptor.xml and put it
under \texttt{src/main/resources/}}, so that we could easily find the entry point
of your pipeline and locate all your component descriptors and thus your
component implementations.
To import another descriptor, you should use the following include directive (don't
confuse it with include/href and import/name!):
\begin{verbatim}
<import location="..."  />
\end{verbatim}
Note that in the case of import/location
names of imported descriptors are relative to the folder of the importing descriptor.

\end{enumerate}

\subsection{Collection reader}

\begin{enumerate}

\item Since your pipeline is required to take an arbitrary file (with the same
format as \texttt{sample.in}) as input, you should define a configuration
parameter for the input file path in the collection reader. It could be
``src/main/resources/data/sample.in'' (without quotes) for your own test, but
\textbf{in your final submission, please put ``hw1.in'' (without quotes) as the
value for the parameter}, which is the path of the file used to test your
pipeline. This file should be read from the file system, not as a CLASSPATH resource.
In \S~\ref{SectionPathGuide} we discuss all path-related issues
in details.

\item Ideally, depending on where the input file is located (on a local file
system, in a jar file, or from an external sources, e.g. network), your
collection reader needs to be general enough to establish a connection to the
file source and open a stream to read the content. But for our task, you only
need to consider a file located on the file system, which means the basic
\texttt{FileInputStream} and \texttt{FileReader} will do the work.

\end{enumerate}

\subsection{Cas consumer}

\begin{enumerate}
  
\item Please refer to a prior section to find out the expected output format.

\item Similar to the collection reader, you should define a configuration
parameter for the output file path. \textbf{Please name it as ``hw1-ID.out''
(without quotes) in your final submission, where ID is again your Andrew ID},
and then your cas consumer will write the output file to the project root
directly.

\item You could create multiple views for a CAS, e.g., one for all the input
elements, and one for all the intermediate results, and one for the final
annotations, then your cas consumer can read the document ID from the first
view, and the annotations from the last view to produce the output file. If you
don't do it in this way, then you should probably include the document ID as a
feature in all the intermediate types and final annotation type.

\item You can also consider your evaluators (which can be
Precision/Recall/F-measure, etc.) as cas consumers in your preliminary
experiments.

\end{enumerate}

\subsection{Annotator}

\begin{enumerate}

\item If you want to employ a resource (e.g., a model you trained offline) 
Be sure to put your resource in \texttt{src/main/resources} so that your
submission will also bundle those resources along with your code.
Please check \S~\ref{SectionPathGuide} for more details related
on file locations.

%in your annotator, you could consider UIMA's \emph{resource manager} (refer to the
%official tutorial for details about this).

\item \textbf{If you want to incorporate other NLP or machine learning tools,
please try to avoid non-Java packages. If you want your artifact to depend on
Java packages other than those provided by the archetype but can be found
in the course repository, you can add a Maven dependency for this artifact, if
you have the jar package on your machine, but it doesn't exist in the course
repository, please let us know, we will try to deploy them in a 3rd party
repository.}

\item Remember to add comments and Javadocs to your annotators, we will also
evaluate the quality of your code.

\item The most creative ideas will happen in the intermediate annotators. We
have mentioned some possible solutions to do this. You can try and implement
multiple approaches, and annotate the gene mentions for the sentence multiple
times. All the annotations should be kept in the CAS until a final ``merging''
component takes all the annotations and make a final judgment.

You can use the type's \emph{source} feature to identify, among all the
approaches, which annotator is attributed this annotation, and \emph{confidence}
feature as a weight to vote for a final decision. This approach was also applied
in IBM's Watson system.

\end{enumerate}

\section{A hopefully ultimate guide to paths, resources, XML descriptors, and pesky leading slashes
in resource names}
\label{SectionPathGuide}
In this section,
we summarize all the path-related information that is necessary to create a successfully working submission. To test if your jar is, indeed, working, you can use \href{https://github.com/amaiberg/software-engineering-preliminary/blob/master/grading_hw1_2}{the grading script (clickable link)}. 
It should now work on Linux and Mac.

Note that we have files of three types:
\begin{itemize}
\item Regular files from the file system. In this assignment, these are only input/output files.
\item Your resource files. They are located in the JAR-file. Must be placed into src/main/resources or a subfolder thereof.
\item XML descriptor files. Must be placed into \url{src/main/resources} or a subfolder thereof.
When you release your code (or do \texttt{mvn install}) these files are also included into the JAR-file.
\end{itemize}
 
Note that the grading script takes into account both detected mentions and their offsets, if your annotator produces correct strings, but wrong offsets, this doesn't count as the right answer.


\subsection*{INPUT-OUTPUT FILES}
To simplify things, all submissions should read hw1.in as input and produce hw1-ID.out as output. Please, no directory prefixes: you read a file from the current directory and save another one to the same current directory. ID is your Andrew/SCS id. 

\subsection*{RESOURCE-FILES}
\subsubsection*{Lingpipe models}

Place a model to src/main/resources or a subfolder thereof. Use the following function to read the model:
\href{http://alias-i.com/lingpipe/docs/api/com/aliasi/util/AbstractExternalizable.html#readResourceObject%28java.lang.Class,%20java.lang.String%29}{readResourceObject(Class\textless ?\textgreater clazz, String resourcePathName)}
resorucePathName should start with a slash! 
 
\paragraph{Example1.} Your model file is \url{src/main/resources/model.bin} and the resourcePathName should be \url{/model.bin}. 
\paragraph{Example2.} Your model file is \url{src/main/resources/some_sub_dir/model.bin} and the resourcePathName should be \url{/some_sub_dir/model.bin}.

\subsubsection*{Custom dictionary}
Again, place a model to \url{src/main/resources} or a subfolder thereof. 
Open resource as a stream. We created \href{https://github.com/amaiberg/software-engineering-preliminary/tree/master/read_resource_example}{a working example demonstrating this technique (clickable link)}. 
Depending on the function that you use, you may or may not need to specify the leading slash! See, the example for details.

\subsubsection*{Some other NER or NLP tool}

If another NER can read a model from the Java InputStream, you are good. 
See the previous subsection for instructions of opening a stream. 
If the NER can't read models from the stream, you can't employ such a NER.

\subsection*{UIMA XML DESCRIPTORS}
Again, place descriptors to 
\url{src/main/resources} or a subfolder thereof.

The main descriptor CpeDescriptor.xml (note that names are case-sensitive) should be placed directly to src/main/resources. 
To import another descriptor, you should use the following include directive (don't
confuse it with include/href and import/name!):
\begin{verbatim}
<import location="..."  />
\end{verbatim}
Note that in the case of import/location
names of imported descriptors are relative to the folder of the importing descriptor.

\paragraph{Example} You have a subfolder \url{descriptors} and two XML descriptors:
\begin{itemize}
\item \texttt{descriptors/a.xml}
\item \texttt{descriptors/b.xml}
\end{itemize}

To import \texttt{b.xml} from \texttt{a.xml} use:
\begin{verbatim}
<import location="b.xml" />
\end{verbatim}



\section{Final report}
There is no template for the report, but we expect you to discuss 
\textbf{both}
the architectural (UML, type system, engineering, design
pattern, etc.) and algorithmic aspects (knowledge sources, NLP tools,
machine learning methods, etc.) of your system. 
In addition,
you need to describe the preliminary experiments you
conducted to evaluate the overall performance of your system.
If you implemented several solutions (or several variants of
 tagging components), 
it will be great to evaluate their performance as well.

When writing the report,
please try to answer the following questions (from the official Gene Mention task submission
guideline):

\begin{enumerate}
\item Please identify/describe any machine learning techniques used:.......... 
\item Please identify/describe any NLP techniques/components used:........ 
\item Please identify/describe any external (marked up text) training data used:......... 
\item Please identify/describe any external lexical resources (terminology lists)used:........ 
\item Please describe any rule sets used:......... 
\item If your system interacts with or uses data from any biological database(s), please describe:.......... 
\item Please identify/describe any other relevant resources used to train/develop your system:......... 
\item Please describe the general data flow in your system:.......... 
\item Other information of interest:.........
\end{enumerate}

Finally, \textbf{don't forget to put your name and Andrew ID at the top of the document},
name the file as ``hw1-ID-report.pdf'' and put it under
\texttt{src/main/resources/docs}.



\newpage
\section*{Proposal Outline}
\begin{enumerate}
\item What questions will you focus on for your final system (exact answer generation part)?
\begin{itemize}
\item Yes/No questions
\item Factoid questions
\item List questions
\item Summary questions
\end{itemize}

\item What technologies/ 3rd party tools do you plan to use?

\item Brief overview of the main components of the system.
        What is the ‘Intelligent’ component in your system?

\item Critical aspects of the system/ Possible pitfalls in your system.

\item Evaluation/Error analysis of your system: a brief statement explaining how you will evaluate your
method(s).

\item Team members and their andrew ids.
\end{enumerate}

\end{document}
