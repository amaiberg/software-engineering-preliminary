\section{Creating Maven project from the archetype}
For this homework, we create another archetype called \texttt{DEIIS-project-archetype} to help you quickly get your development started. We briefly show you the process you've gone through for your Homework 1.

\begin{enumerate}

\item Open your Eclipse's \textbf{Preferences} window, and navigate to
\textbf{Maven} $\rightarrow$ \textbf{Archetypes}, and click \textbf{Add Remote
Catalog\ldots}.

\item Type the following URL into the \textbf{Catalog File} field.
\url{https://raw.githubusercontent.com/oaqa/DEIIS-project-archetype/master/archetype-catalog.xml}
\normalsize

Optionally, you can add a \textbf{Description} for this catalog, for example
``BioQA Catalog''. Then click \textbf{OK} on the \textbf{Remote Archetype
Catalog} window and another \textbf{OK} on the \textbf{Preferences} window.
\item Add the following to your \verb|settings.xml| from Listing \ref{settings}.

\lstinputlisting[language=XML,linewidth=1.1\textwidth,caption=Configuring settings.xml,label=settings]{settings.xml}

\item Now you can follow almost follow the same steps to import to Eclipse as you did
for Homework 1. Since we have created the archetype for you, remember to
unselect \textbf{Create a simple project (skip archetype selection)}. Then click
\textbf{Next}.
 
\item Here you can select ``BioQA Catalog'' (or other names you specified
in the previous step) or ``All Catalogs'' in the drop-down menu for
\textbf{Catalog}. Then, type in ``project-archetype'' (without quotes) in the \textbf{Filter}
field, and in order to get the latest snapshot archetypes, you need to check \textbf{Include snapshot archetypes} as well. Select the archetype listed below, and click next to continue.

\item In the next window, you are asked to specify the \textbf{Group Id} and
\textbf{Artifact Id}. Similar to Homework 0 and 1, the Group Id is

\begin{center}
\textbf{edu.cmu.lti.11791.f14.project}
\end{center}

and Artifact Id is

\begin{center}
\textbf{project-teamXX}
\end{center}

with XX being your team number. Remember to specify \texttt{Package} as

\begin{center}
\textbf{edu.cmu.lti.11791.f14.project}
\end{center}

Then click \textbf{Finish}.


\item You need to edit the \texttt{pom.xml} file to type in the SCM information
of your GitHub repository for project as you did in Homework 0.

\item Same as before, you probably need to right-click the project name, and
click \textbf{Maven} $\rightarrow$ \textbf{Update Project} to download the
dependencies.

\end{enumerate}

You can see that we have included:

\begin{itemize}

\item the \texttt{pom.xml}. To see the dependencies you can click the \textbf{Dependencies} tab after you double-click the pom file. There you will see that in addition to the core UIMA components, the project depends only on the \texttt{bioasq-gopubmed-client} project. This client will simplify calling all the ontology services defined for this milestone.

If you want to take a look at what is inside \texttt{bioasq-gopubmed-client} project and
other indirect dependencies, you can unfold the \texttt{Maven Dependencies}
folder under your project name in the Package Explorer View.
\end{itemize}

Before you commit and push all the initial code changes to GitHub repository, we
suggest you to first test if you can successfully run the pipeline.
\begin{comment}

\begin{enumerate}
item Right-click this file in the Package Explorer View, and then click
\textbf{Run As} $\rightarrow$ \textbf{hellobioqa}. Wait until all 28 questions
are processed, and the evaluation results are printed to the console, and you
find no exceptions are thrown.
\item Now, other team members are able to clone the repository to their workspaces and start working on particular task. When you are asked to \textbf{Select a wizard to use for importing projects}, don't forget to select \textbf{Import existing projects} as long as your team leader commited .project and .classth to the repository.
\end{enumerate}
\end{comment}
