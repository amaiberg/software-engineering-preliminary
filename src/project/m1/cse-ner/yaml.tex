
\section{Writing YAML descriptor}

To write a descriptor for an annotator in CSE is much easier than you did in
Homework 1 since it is based on YAML instead of XML. If you want to find a tool
that can assist you to edit YAML files, e.g., syntax hightlight, automatic YAML
grammar checking, you can consider Eclipse YEdit plug-in.

\begin{enumerate}

\item Under \texttt{src/main/resources/hellobioqa} directory, create a
subdirectory \texttt{teamXX} for your components, and then probably you want to
create different subdirectories (\texttt{keyterm}, \texttt{retrieval}, and
\texttt{passage}) in \texttt{teamXX} directory for different phases.

\item Then create the descriptor file (with extension name \texttt{yaml}) for
your component. Open the descriptor file. If you installed YEdit plug-in, then you are
able to right-click the file, and click \textbf{Open With} $\rightarrow$
\textbf{YEdit YAML Editor}.

\item If you component doesn't required any additional configuration parameters,
then just simply type your class path and a \verb|persistence-provider| into
your file, like in Listing \ref{lst:simplest}. Remember that the
\verb|persistence-provider| is required by all the components. For components
with parameters, you can declare the parameters and values like in List
\ref{lst:simpler}. If you want to set different values to the same parameter,
and test their performances in a single experiment, you can use
\verb|cross-opts| like in Listing \ref{lst:simple}.

\lstinputlisting[language=Python,float,linewidth=1.1\textwidth,caption=Descriptor
for a component without parameters,label=lst:simplest]{simplest.yaml}

\lstinputlisting[language=Python,float,linewidth=1.1\textwidth,caption=Descriptor
for a component with parameters,label=lst:simpler]{simpler.yaml}

\lstinputlisting[language=Python,float,linewidth=1.1\textwidth,caption=Descriptor
for a component with cross-opts,label=lst:simple]{simple.yaml}

\item Finally, you need to properly set the pointers to all of them in the main
yaml. Open the \texttt{src/main/resources/hellobioqa/hellobioqa.yaml}, and
change the \texttt{author} to ``teamXX''. Add your components into the
\verb|options| fields of the first phase, to make the \verb|pipeline| field look
like Listing \ref{lst:hellobioqa}.

\lstinputlisting[language=Python,float,linewidth=1.1\textwidth,caption=Pipeline
descriptor,label=lst:hellobioqa]{hellobioqa.yaml}

\item Run the pipeline again by right-clicking the launch file as we mentioned
earlier. Then git-commit, git-push, git-checkout (change to \texttt{master}
branch. Not sure? Take a look at the Pro Git Book), git-merge, maven
release:prepare, maven release:perform.

\end{enumerate}