% 
% This work is licensed under the Creative Commons Attribution-ShareAlike 3.0
% Unported License. To view a copy of this license, visit
% http://creativecommons.org/licenses/by-sa/3.0/.
%
% author: Eric Nyberg
%
\documentclass[oneside,11pt]{memoir}

%\renewcommand{\chaptername}{Task}

\usepackage{times}
\usepackage{setspace}
%\onehalfspacing
 
\usepackage{titlesec}
\titleformat{\chapter}{\normalfont\Large\bfseries}{Task \thechapter}{1em}{}
\titleformat{\section}{\normalfont\Large\bfseries}{\thesection}{1em}{}
\titleformat{\subsection}{\normalfont\large\bfseries}{\thesubsection}{1em}{}

\usepackage{url}
\usepackage{hyperref}

\usepackage{graphicx}

\usepackage{pstricks}
\usepackage{epsfig}


\usepackage{listings}
\usepackage{color}
\definecolor{dkgreen}{rgb}{0,0.6,0}
\definecolor{gray}{rgb}{0.5,0.5,0.5} 
\definecolor{lightblue}{rgb}{0.95,0.95,1}
\definecolor{mauve}{rgb}{0.58,0,0.82}
\usepackage{multirow}

\definecolor{shadecolor}{gray}{0.9}

\newenvironment{qa}
{\begin{shaded}\begin{itemize}}
{\end{itemize}\end{shaded}}

\title{{\bfseries 11-791 Design and Engineering of Intelligent Information
System \& \\11-693 Software Methods for Biotechnology \\Homework 3}\\
\vspace{1em}
Engineering and Error Analysis with UIMA
\\
{\small This version was built on \today.}
}
\itshape\rmfamily 


\date{}


\begin{document}

\maketitle

\hspace{-0.1\textwidth}
\begin{minipage}{1.2\textwidth}
\vspace{-5em}
\textbf{Important dates}
\begin{itemize}

\item \textbf{Out: October 11.} 

\item \textbf{Due: October 21 (midnight).} 

\end{itemize}

\end{minipage}

\begin{enumerate}

\item \textbf{Submission:} 
Use the same process as for previous assignments (set up GitHub repo, create a Maven 
project, write your code, submit to Maven repo), except that the name has changed to:
hw3-ID. Note that you submit only Task 1 code.
However, your report should describe error analysis and experiments 
carried out in Task 2.

\item \textbf{Write a design report:}
We expect you to document the design of the system that improves
over the system developed in Task 1.
The report should also include the error analysis and 
description of experiments carried out in Task~2.
We will extract the report from the submitted jar file. 
Remember to include your ID as part of the report file name 
as well as of the report itself (at the top of document).
Please submit in PDF format only.
The report should be put in the folder \textbf{src/main/resources/doc}.

%One very special descriptor is the \verb|CpeDescriptor.xml|, which is the entry
%point for the entire pipeline. You are required to name your CPE descriptor
%exactly the same way, and place it in the right place for us to evaluate your
%code.

\item \textbf{Javadocs:} 
Please remember to give an appropriate description (Javadoc comments) for each 
piece of code you create and/or modify.

\item \textbf{Grading:}
This document has two 50pts sections.
For Section \S~\ref{SectionBuildVS}, you will receive at least 20pts,
if your implementation works correctly: i.e., it produces
the output in the right format and saves
the report file with the right name.
If the computed MMR and/or cosine similarity values are slightly incorrect,
no more than 5pts will be detracted.
10pts will be detracted if the produced values do make sense, but often are incorrect
and the MRR is twice is large or small.
If the output file is not generated and contains complete gibberish, 20pts will be
detracted.
The distribution of 30pts will depend on the quality of your implementation
(i.e., how well your code is documented and written).

For Section \S~\ref{SectionError}, 50pts will be given
to high-quality, creative reports.
There will be \textbf{bonus 20pts} for implementation and description
of interesting similarity functions.


\end{enumerate}

\textbf{Useful information}
\begin{enumerate}
\item We encourage you to start Homework 3 as early as possible.

\item 
We expect that most of the general communication between the instructor team and students will take place on 
\href{https://piazza.com/class/hyvsubeilei6dd}{Piazza}.
For private questions, e.g., regarding grades, you may contact instructors by e-mail.
Your friendly TAs are:
\begin{itemize}
\item Avner Maiberg (\href{mailto:amaiberg@andrew.cmu.edu}{\nolinkurl{amaiberg@andrew.cmu.edu}}) 
\item Parag Argawal (\href{mailto:paraga@andrew.cmu.edu}{\nolinkurl{paraga@andrew.cmu.edu}}) 
\item Leonid (Leo) Boytsov (\href{mailto:srchvrs@cmu.edu}{\nolinkurl{srchvrs@cmu.edu}})
\item Xuezi (Manfred) Zhang (\href{mailto:xueziz@andrew.cmu.edu}{\nolinkurl{xueziz@andrew.cmu.edu}}) 
\end{itemize}

\item Source files and pdf file of this assignment are publicly available on 
the  \href{http://github.com/amaiberg/software-engineering-preliminary}{GitHub page}.
Please feel free to fork the project and modify if necessary.
Should you notice any error, you can fix it yourself and create/send a pull request (as some
of you did for previous frameworks).  

\end{enumerate}
\hspace{-0.1\textwidth}

\chapter*{Getting started: Creating Maven project from the archetype}
For this task, we have prepared another archetype to help you quickly build your project. 
The tutorial for Homework 1 might help you create a Maven project from an archetype.
For this Homework, you need to add the new Catalog:
\url{https://raw.githubusercontent.com/oaqa/DEIIS-hw3-archetype/master/archetype-catalog.xml}

The archetype for Homework 3 is:

\begin{center}\texttt{hw3-archetype}\end{center}

Also remember that the Group Id and Artifact Id for Homework 3 are

\begin{center}
\url{edu.cmu.lti.11791.f14.hw3}
\end{center}

and 

\begin{center}\textbf{hw3-ID }\end{center}
with ID being your Andrew Id.
Similarly, you need to edit the pom.xml file to type in the SCM information of
your GitHub repository. 

Unlike other assignments, 
we include a minimal set of descriptors, types, and Java source files.
This set of files defines a complete pipeline. 
The code should compile and run, but it does nothing useful, yet.
Thus, you are not required
to create additional Java files, descriptors, and or types.
However, additional resources (including UIMA types 
and AE descriptors) and source files can be created if needed.

Another important difference is that there is no collection process engine (CPE) descriptor provided.
There is only an aggregate analysis engine (AE) descriptor that imports descriptors of several annotators.
This AE is executed programmatically via a Java file (see section~\ref{SectionAe} for more details):

\begin{center}\url{src/main/java/edu/cmu/lti/f14/hw3/hw3_ID/VectorSpaceRetrieval.java} \end{center}

%\newpage
\chapter{Building Vector space Retrieval Model using UIMA (50pts)}
\label{SectionBuildVS}

The purpose of this assignment is to design a simple vector space retrieval system using the UIMA
framework. 
In the vector space model,
both the query and documents are sparse term vectors.
Given a query vector and a set of short documents, 
we compute the cosine similarity values between the query vector and document vectors.
Then, all documents are ranked in the decreasing order of the cosine similarity.

Note that term frequency vectors should be \textbf{length-normalized}, i.e., their
Euclidean norm is equal to 1. For details, you may read
sections from Chris Manning's et~al book on \href{http://nlp.stanford.edu/IR-book/html/htmledition/term-frequency-and-weighting-1.html#sec:secbagofwords}{term vectors (clickable link)} and \href{http://nlp.stanford.edu/IR-book/html/htmledition/the-vector-space-model-for-scoring-1.html}{vector-space models (clickable-link)}.

The value of the cosine similarity does depend on the \textbf{tokenization algorithm}.
For Task 1, you \textbf{must} use a simple white-space tokenization algorithm,
which does not split on punctuation and considers a sequence of 
adjacent white-spaces to be a single separator.
It also  does not lowercase.
In the archetype, you already have an implementation of such a tokenizer:
please, use the function:
\begin{center}
\textbf{DocumentVectorAnnotator.tokenize0}.
\end{center}

Your goal is to implement the simplest vector space retrieval system
and evaluate its performance using the Mean Reciprocal Rank (MRR).  
The mean reciprocal rank is an average reciprocal rank
computed over a set of queries $\{Q_i\}$:

\begin{equation}\label{EqMRR}
\mbox{MRR} = \frac{1}{|Q|}\sum\limits_{i=1}^{|Q|} \frac{1} {\mbox{rank}_i},
\end{equation}
where $\mbox{rank}_i$ is a position (i.e., rank) of the first relevant document
in the list of retrieved documents (sorted in the decreasing order of the cosine similarity with respect to a query number $i$).

Consider an example, where the three most highly-ranked documents are considered 
non-relevant, while the fourth one is considered to be relevant. 
Then, the reciprocal rank for this query is equal to $1/4$.

If there are ties, i.e., several documents have the same
similarity score,
the ranking of document cannot be defined unambiguously.
For the purpose of this assignment only, if you have ties,
you need to rank relevant documents higher than not relevant. 
For all other documents, ties can be broken arbitrarily.
In this assignment there is only \textbf{one relevant document} per query.
Hence, there is no ambiguity in computing the MRR and generating the output report.


The code pipeline should work without any compilation errors but it does not produce any relevant results or evaluation measurements.
To run the pipeline from the command line (terminal window), you can type the following
(note that ID is your SCS/Andrew ID; the dash was automatically converted to the underscore when the project was created from the archetype):

\begin{center}\small\textbf{mvn exec:java 
       -Dexec.mainClass=\url{edu.cmu.lti.f14.hw3.hw3_ID.VectorSpaceRetrieval}}\end{center}

We have already given comments inside the code (with //TO DO) to indicate what is recommended
to implement. Feel free to deviate from these recommendations
 or reorganize the code, if this is necessary.
The layout of the provided template code is as follows:

\vspace{2em}
\hspace{-3em}
\begin{minipage}{0.9\textwidth}
\footnotesize
\begin{verbatim}
hw3-ID
|- pom.xml
`- src
     `- main
      |- java/**/*.java        /* Java code */
     |    `src/main/java/edu/cmu/lti/
     |       `f14/hw3/hw3_ID/
     |                   `VectorSpaceRetrieval.java /* In particular:
     |                                                main Java class */
     |                           
     `- resources
       |- data/documents.txt  /* toy data */ 
       |- descriptors/*.xml   /* all pipeline and typesystem descriptors */
       |     `retrievalsystem
       |        `VectorSpaceRetrieval.xml  /* In particular: 
       |                                      the main AAE descriptor */
       `- doc
            `- hw3-ID-report.pdf /* your report */

\end{verbatim}
\normalsize
\end{minipage}



\section{Input}
The text collection for this homework is given in documents.txt file in 
the \textbf{src/main/resource/data} directory of the project. 
For simplicity we use small documents each of which consists of only one sentence.
Thus, there is no need to support multiple sentences in the document for this
homework.
Sentences occasionally miss punctuation or have minor spelling/grammar error:
most of them were left on purpose.

A sample file is outline below (fields are tab-separated):

\begin{verbatim}
qid=1 rel=99 In which year did a purchase of Alaska happen?
qid=2 rel=1 Alaska was purchased from Russia in year 1867.
qid=2 rel=1 The purchase of Alaska took place in 1867.
qid=2 rel=99 ...
qid=2 rel=1 ...
qid=2 rel=0 ...
\end{verbatim}

Each line has a query id (prefixed by \texttt{qid=}), 
a binary relevance assessment (prefixed by \texttt{rel=}), 
and a document text (which is only one sentence!). 
For documents assessed as relevant, we have \texttt{rel=1},
while \texttt{rel=0} means that a document is not considered to be relevant.
There is only \textbf{one relevant} document per query.

If a line contains a special marker \texttt{rel=99},
the line represents a query rather than a document.
For example, ``In which year did a purchase of Alaska happen?
'' means that there is a query/question ``In which year did a purchase of Alaska happen?'' 
that has the id one.

All lines are sorted by the question id (qid).
The query sentence with qid=$x$ always precedes
document sentences with the same qid=$x.

\section{Analysis Engine}
\label{SectionAe}
We provide you with a simplified aggregate analysis engine AAE that runs without 
a collection processing engine descriptor (CPE).
This AAE is executed programmatically via the following Java file:

\begin{center}\url{src/main/java/edu/cmu/lti/f14/hw3/hw3_ID/VectorSpaceRetrieval.java} \end{center}

The location of this AAE descriptor is:

\begin{center}\url{src/main/resources/descriptors/retrievalsystem/VectorSpaceRetrieval.xml} \end{center}

The AAE contains three primitive analysis engines:
\begin{enumerate}
\item DocumentReader
\item DocumentVectorAnnotator
\item RetrievalEvaluator
\end{enumerate}

The DocumentReader parses lines from the input document file and saves the result of the parse (query id, relevance assessment, and document text) in a CAS.

The DocumentVectorAnnotator is a stub for the annotator that creates sparse term vectors
(recall that a document term vector contains the number of occurrences (i.e., a frequency)
for each word appearing in a document). 
You need to finish the implementation (read the comments/code and check for //TO DO markers).


The RetrievalEvaluator is a stub for an evaluation component. 
You need to finish the implementation
(again, read the comments/code and check for //TO DO markers).
This evaluation component needs to do the following:
\begin{enumerate}
\item Compute the cosine similarity between the query sentence and each of the subsequent
document sentence. Remember to use the
simple white-space tokenizer, e.g.,
provided by the function \textbf{DocumentVectorAnnotator.tokenize0}.
\item Using provided relevance assessments, compute a reciprocal rank for each query.
Then, these reciprocal ranks are averaged to obtain the Mean Reciprocal Rank (see Eq.~\ref{EqMRR} for details).
\item Create a report and save it to the file \textbf{report.txt} (just report.txt, please). 
For each query, your code has to select a \textbf{relevant} document with the
\textbf{highest} cosine-similarity score and print the following:
\begin{verbatim}
cosine=...<tab>rank=...<tab>qid=...<tab>rel=1<tab><sentence>
\end{verbatim}
In the end of the report, you \textbf{must} print the compute MRR. e.g:
\begin{verbatim}
MRR=0.75
\end{verbatim}
\item Round all floating-point numbers to have \textbf{four} significant digits
and use \textbf{double} precision floating point numbers (i.e., the type \textbf{double}).
Sort all sentences by the query id.
Your sample report file may look as follows (these scores are for illustrative purposes only 
and may not be true, remember that fields are separated by \textbf{tabs}):
\begin{verbatim}
cosine=0.5431 rank=1 qid=1 rel=1 Alaska was purchased ...
cosine=0.1123 rank=2 qid=2 rel=1 The purchase of Alaska ...
MRR=0.75
\end{verbatim}
\item To verify correctness of your implementation,
\href{https://raw.githubusercontent.com/amaiberg/software-engineering-preliminary/master/src/retrieval-error-analysis/sample_report1_5.txt}{we provide an output} (clickable link) produced
by a reference implementation.
\end{enumerate}

\begingroup
\renewcommand{\cleardoublepage}{}
\renewcommand{\clearpage}{}
\subsection{How to aggregate data coming from different CASes} 
\endgroup

To evaluate performance you may need to aggregate information
supplied with different CASes.
Because the pipeline processes one CAS at a time,
this can be problematic unless you persist intermediate results.

UIMA manual discourages us from persisting data \textbf{in class variables} (use 
of static variables is discouraged as well),
because each CAS might be processed by a separate Java virtual machine.
However, for the purposes of this assignment, one can store data in memory.
Alternatively, you can you use a file system.
%\newpage

Persisting to memory or file would work for our assignment, 
because the pipeline executes the components on the same machine 
and all the code runs within a single Java virtual machine.
In a world of distributed processing, however, you may want to store such information in a database,
on a network drive, or employ some other machine-independent storage.

\textbf{Important catch:} Don't try to store references to classes inherited from the
UIMA class \textbf{Annotation}.

\vspace{2em}
\section{Type System}\label{SectionType}
The type system currently has two types: ``Document'' and ``Token''.
Feel free to extend this type system if necessary.
At present, the type Document contains the following information: Relevance Assessment (called relevanceValue), Query ID, Text String, Token
List. The Relevance Value, query ID, Text String are extracted from the text collection. 

A list (\textbf{uima.cas.FSList}) of tokens 
represents a ``bag-of-words'' feature vector
that has to be constructed for the retrieval system. 
To manipulate objects of the type \textbf{uima.cas.FSList},
you can use the provided class \textbf{utils.Utils}.

\textbf{Hint1}:
one can generate type system classes from the type system XML descriptor via the UIMA JCasGen feature).

\textbf{Hint2:} tokenList feature of Document needs to be populated by your annotators. 
A Token contains term and its corresponding frequency (in a document). 
You will need to implement the code that populates these sparse frequency vectors.


%\begingroup
%\renewcommand{\cleardoublepage}{}
%\renewcommand{\clearpage}{}
\chapter{Error Analysis (50pts + 20pts bonus)}
\label{SectionError}
%\endgroup
The goal of this Task is demonstrate how a choice of tokenizing, stemming, and/or ranking
algorithm affects performance.

In Task 1, you employed a basic-white space tokenization algorithm.
Carry out an error analysis to identify situations where the
combination of this tokenization algorithm with the cosine similarity failed.
\textbf{Classify errors} into classes: provide statistics
by error types, e.g., in a tabular format.
For example, you can report that for 10 queries, the most relevant answers
were not selected due to the vocabulary mismatch. In this example,
``vocabulary mismatch'' is a type of error.

After carrying out error analysis,
you are likely to have ideas about improving performance.
In this task, you need to implement some of these ideas.
You may choose to employ:

\begin{itemize}
\item Better tokenization algorithms;
\item Better stemming algorithm: The simple vector model does not understand 
that \textbf{die} and \textbf{died} denote the same thing in most contexts.
You can use the provided wrapper for the Stanford lemmatizer (package \textbf{util.StanfordLemmatizer})
or any other stemmer/lemmatizer;
\item Better or different similarity measures: unnormalized TF-IDF,
the \href{http://en.wikipedia.org/wiki/S%C3%B8rensen%E2%80%93Dice_coefficient}{Dice coefficient (clickable link}), the \href{http://en.wikipedia.org/wiki/Jaccard_index}{Jaccard coefficient (clickable link)},
or even \href{http://en.wikipedia.org/wiki/Okapi_BM25}{BM25 (clickable link)}.
\end{itemize}
Again, remember, that you can change the system arbitrarily.

Note that there will be \textbf{bonus 20pts} for implementation and description
of interesting similarity functions. 
To get these bonus points, you need to 
implement these functions, compare them against the cosine similarity and each other,
\textbf{as well as to describe} your design and experiments in the report.
If you get an improvement, it is not a bad idea to compute p-values,
confidence intervals, or apply other methods to evaluate statistical significance (i.e.,
to answer a question: can these improvements be due to random factors?).

Your work on Task 2 should be described in a report, 
which is an \textbf{important part} of your submission: 
allocate sufficient time to do a thorough job. 
Feel free to consult the TAs if you have questions regarding the report.

The report can be in a free form (you do not have to follow a pre-specified template).
Yet, we expect you to consider the system design from several architectural aspects (UML, type
system, engineering, design pattern, etc.) and several algorithmic aspects (knowledge sources, NLP
tools, machine learning methods, etc.), documenting what you feel is best to communicate
the essence of your design. 

For example, it is worth drawing a UML diagram to represent a novel design pattern.
However, please, do not draw diagrams
for well-known design pattern (e.g. Template Method), use a brief description instead.

Finally, do not forget to put
your name and Andrew ID at the top of the document, 
name the file as \textbf{hw3-ID-report.pdf} and put it
under \textbf{src/main/resources/doc}. 
Do not submit the implementation for Task 2, only the report.

%% This prevents a nasty page break!!!
%\begingroup
%\renewcommand{\cleardoublepage}{}
%\renewcommand{\clearpage}{}
%\chapter*{Bonus (20pts)}
%\endgroup

%Try other similarity measures instead of cosine similarity for ranking (i.e. dice coefficient, Jaccard
%coefficient) and include a comparison in your report.


%\textbf{Hints:} There is no preferred setup for this assignment, so you may write a new annotator to accomplish this
%task, thereby demonstrating your understanding of the UIMA framework and Java programming.
%
%\begin{enumerate}
%\item You are allowed to modify the provided type system;
%\item You are allowed to modify the provided classes and annotators. You are also allowed to create new
%methods, a new class, or even a new package.;
%\item If you can think of efficient design patterns for comparing multiple document ranking strategies in a
%single execution of the pipeline, feel free to implement!
%\end{enumerate}

\end{document}
