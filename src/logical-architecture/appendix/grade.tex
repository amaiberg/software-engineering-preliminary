
\subsection{Grading Guidelines}

\subsubsection{Designing and Implementing UIMA Analysis Engine (60 pts)}

\begin{itemize}

\item Basic requirement: the system should work (20pts). 
If the system does not work or misconfigured you may lose points.
For example, you will lose a few points if we need to edit your descriptors
\textbf{to specify the correct input file}.
If the system does not work and we cannot make it running (\textbf{we do 
not have to try very hard!}), you will get \textbf{zero} points.

\item Additional 10 pts will be given for interesting design solutions.
For example, you may get extra credit for an interesting type system,
which, e.g., employs inheritance. 
Beware not to over-engineer! You can be penalized for introducing
unnecessary complexity.

\item Additional 30 pts can be earned for performance of your system. 
The f-score of 0.7 will earn you 30pts.
If you get a lower or higher score, the nominal score of 30 pts will be scaled linearly (e.g.,
15 for the f-score of 0.35).
It is possible to get more than 30 pts here, but we do not many students
to perform this well.

\end{itemize}

\subsubsection{Documentations, comments, coding style (30 pts)}

We will check a completeness of your report (you should provide
all necessary examples/explanations and address all our questions)
as well as the quality of your Javadocs (need to provide sufficient
documentation for users).

\begin{itemize}

\item A basic documentation and report get maximum 20 pts.
Additional 10 pts will be given for extra quality (surprise us).
Some points will be subtracted if, e.g., Java docs are missing.

\item To get the maximum score of 30 pts one needs to
excel at the following:

\begin{enumerate}
\item discussing of design aspects of the problem;
\item discussing of algorithmic aspects of the problem, in particular,
at comparing several alternative solutions;
\item applying interesting yet simple design patterns (over-engineering is an evil!);
\item providing concise yet complete documentation;
\item providing other insights not covered by the previous items.
\end{enumerate}

\end{itemize}

\subsubsection{Submission, folder structure, and name convention (10 pts)}
We will check the correctness of your submission, folder structure,
and the code.

\begin{itemize}

\item We expect your homework to be located at the right repository 
and your project folders/files be organized as outlined in
the beginning of the assignment (5 pts). 

\item The names of your types and files should follow Java naming convention (5 pts). 
See, e.g., this document for details \href{http://java.about.com/od/javasyntax/a/nameconventions.htm}{on naming conventions.}


\end{itemize}
